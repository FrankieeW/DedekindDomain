\documentclass[11pt]{article}
\usepackage{blueprint}

\title{DedekindDomain Blueprint}
\author{Formalization plan for George Boxer Notes (Lectures 10--11)}

\begin{document}
\maketitle
\tableofcontents

\section{Scope}
This blueprint covers only Lecture 10 and Lecture 11 of
\emph{George Boxer Notes - NO SOLUTIONS.pdf}:
Euclidean domains (context), Dedekind domains, and unique factorization of nonzero ideals.

\section{Definitions}
\definitions

\begindefinition{Euclidean function}{def-euclidean-function}
Let $R$ be an integral domain. A map $f : R \to \mathbf{N}$ is Euclidean if for all
$a,b \in R$ with $b \neq 0$, there exist $q,r \in R$ such that
$a=bq+r$ and either $r=0$ or $f(r)<f(b)$.
\lean{../../DedekindDomain/Lecture10.lean}
\enddefinition

\begindefinition{Norm Euclidean integer ring}{def-norm-euclidean}
For a number field $K$, the integer ring $\mathcal{O}_K$ is norm Euclidean if $|\mathrm{N}_K|$
is a Euclidean function on $\mathcal{O}_K$.
\uses{def-euclidean-function}
\lean{../../DedekindDomain/Lecture10.lean}
\enddefinition

\begindefinition{Dedekind domain}{def-dedekind-domain}
An integral domain $R$ is Dedekind if:
(1) $R$ is integrally closed;
(2) every nonzero prime ideal is maximal;
(3) $R$ is Noetherian.
\lean{../../DedekindDomain/Lecture10.lean}
\enddefinition

\begindefinition{Noetherian ring conditions}{def-noetherian-conditions}
Equivalent conditions:
(1) ACC on ideals; (2) every nonempty set of ideals has a maximal element;
(3) every ideal is finitely generated.
\lean{../../DedekindDomain/Lecture10.lean}
\enddefinition

\section{Lecture 10 Results}

\beginlemma{Norm Euclidean criterion (L10.3)}{lem-norm-euclidean-criterion}
For number field $K$, $\mathcal{O}_K$ is norm Euclidean iff
for every $\alpha\in K$ there exists $\beta\in \mathcal{O}_K$
with $|\mathrm{N}_K(\alpha-\beta)|<1$.
\uses{def-norm-euclidean}
\lean{../../DedekindDomain/Lecture10.lean}
\endlemma

\beginlemma{Finite index of nonzero ideals in $\mathcal{O}_K$ (L10.11)}{lem-ideal-finite-index}
If $I\subseteq \mathcal{O}_K$ is nonzero, then $[\mathcal{O}_K : I] < \infty$.
\uses{def-noetherian-conditions}
\lean{../../DedekindDomain/Lecture10.lean}
\tangled
\endlemma

\beginproposition{$\mathcal{O}_K$ is Dedekind (L10.10)}{prop-ok-dedekind}
For any number field $K$, $\mathcal{O}_K$ is a Dedekind domain.
\uses{def-dedekind-domain}
\uses{lem-ideal-finite-index}
\lean{../../DedekindDomain/Lecture10.lean}
\tangled
\endproposition

\beginproposition{PID implies Dedekind (L10.12)}{prop-pid-dedekind}
Every PID is a Dedekind domain.
\uses{def-dedekind-domain}
\lean{../../DedekindDomain/Lecture10.lean}
\endproposition

\section{Lecture 11 Results}

\beginlemma{Cayley-Hamilton trick on ideals (L11.3)}{lem-fractional-stabilizer}
If $I\subseteq R$ is an ideal and $x\in \mathrm{Frac}(R)$ satisfies $xI\subseteq I$,
then $x\in R$.
\uses{def-dedekind-domain}
\lean{../../DedekindDomain/Lecture11.lean}
\tangled
\endlemma

\beginlemma{Product of prime ideals inside a nonzero ideal (L11.4)}{lem-prime-product-contained}
For every nonzero ideal $I\subseteq R$, there exist nonzero prime ideals
$\mathfrak p_1,\dots,\mathfrak p_r$ such that
$\mathfrak p_1\cdots\mathfrak p_r\subseteq I$.
\uses{def-noetherian-conditions}
\lean{../../DedekindDomain/Lecture11.lean}
\tangled
\endlemma

\beginlemma{Prime ideal test for products of ideals (L11.5)}{lem-prime-ideal-product}
If $\mathfrak p$ is prime and $IJ\subseteq \mathfrak p$, then
$I\subseteq \mathfrak p$ or $J\subseteq \mathfrak p$.
\lean{../../DedekindDomain/Lecture11.lean}
\endlemma

\beginlemma{Existence of non-integral multiplier (L11.6)}{lem-nonintegral-multiplier}
If $I\subsetneq R$ is nonzero, then there exists $x\in \mathrm{Frac}(R)\setminus R$
such that $xI\subseteq R$.
\uses{lem-prime-product-contained}
\uses{lem-prime-ideal-product}
\uses{def-dedekind-domain}
\lean{../../DedekindDomain/Lecture11.lean}
\tangled
\endlemma

\beginproposition{Invertibility of nonzero ideals (L11.9)}{prop-ideal-invertible}
For nonzero ideal $I\subseteq R$ and $0\neq a\in I$, there exists ideal $J\subseteq R$
with $IJ=(a)$.
\uses{lem-fractional-stabilizer}
\uses{lem-nonintegral-multiplier}
\lean{../../DedekindDomain/Lecture11.lean}
\tangled
\endproposition

\begincorollary{Cancellation of nonzero ideals (L11.10)}{cor-cancellation}
If $I,J,K\subseteq R$ are nonzero ideals and $IJ=IK$, then $J=K$.
\uses{prop-ideal-invertible}
\lean{../../DedekindDomain/Lecture11.lean}
\endcorollary

\begincorollary{Containment equals divisibility (L11.11)}{cor-contain-divide}
For nonzero ideals $I,J\subseteq R$:
$I\subseteq J$ iff there exists ideal $K$ with $I=JK$.
\uses{prop-ideal-invertible}
\uses{cor-cancellation}
\lean{../../DedekindDomain/Lecture11.lean}
\endcorollary

\begintheorem{Unique factorization of nonzero ideals (L11.1)}{thm-ideal-factorization}
If $R$ is Dedekind and $I\subseteq R$ is nonzero, then
$I=\mathfrak p_1\cdots\mathfrak p_r$ for prime ideals, uniquely up to reordering.
\uses{def-dedekind-domain}
\uses{lem-prime-ideal-product}
\uses{cor-cancellation}
\uses{cor-contain-divide}
\lean{../../DedekindDomain/Lecture11.lean}
\tangled
\endtheorem

\section{Milestones}
\beginitemize
\item M1 (foundation): complete all definitions in Lecture 10 and ring-theoretic setup.
\item M2 (number field instance): prove $\mathcal{O}_K$ is Dedekind.
\item M3 (abstract engine): complete L11.3--L11.9.
\item M4 (main theorem): complete L11.1 and refactor dependencies.
\enditemize

\end{document}
